\section{Introduzione}
	\subsection{Analisi d'utenza}
	\par Il sito è rivolto ad utenti in età adulta, intenzionati a vendere un’automobile di loro proprietà od in cerca di un mezzo. Il target è molto ampio in quanto si stima che, in media, in Italia un abitante su due possiede ed utilizza un'autovettura e quindi circa met\`a della popolazione italiana potrebbe essere interessata alla vendita oppure all'acquisto di un veicolo usato. Il pubblico è molto variegato e composto sia da utenti che da sempre hanno navigato sul web sia da persone poco familiari con la tecnologia.
	\subsubsection{Guidatori inesperti}
	\par I guidatori inesperti sono un tipo di utenza giovane, intesi come i neopatentati ed i guidatori che hanno ottenuto la patente di guida da meno di dieci anni. Abituati alla tecnologia, sono alla ricerca di un sito web moderno per poter acquistare la loro prima auto, eseguendo ricerche veloci preferibilemente dal loro dispositivo mobile.
	\subsubsection{Guidatori esperti}
	\par I Guidatori esperti sono un tipo di utenza tendenzialmente poco avvezza alle tecnologie, poiché si riferisce alla fascia d'età delle persone con età maggiore ai 28 anni. Essi comprendono lavoratori e/o pensionati in cerca di un'auto per sé e padri di famiglia in cerca, ad esempio, della prima auto da acquistare o regalare al proprio figlio neopatentato. Costoro effettueranno ricerche ad ampio raggio, poiché vorranno visionare tutte le auto appartenenti ad una determinata classe (per stile, cilindrata, ...) prima di decidere quale acquistare.
	\subsection{Considerazioni finali}
	\par Attraverso un’interfaccia intuitiva ed un linguaggio informale il sito cerca di essere fruibile alla quasi totalità delle persone interessate alla compravendita di automobili. Inoltre si cerca di non allontanare i meno esperti permettendo una ricerca modulare con molti campi opzionali. In questa maniera si spera di non intimidire gli utenti con poca dimestichezza per quanto riguarda le componenti specifiche delle automobili. Questo vale anche per la sezione annunci: è infatti possibile pubblicare un annuncio senza scendere nei dettagli riguardanti le qualità di un veicolo. Il sito dovrà essere responsive, in modo da potersi adattare a ogni tipo di dispositivo, e dovrà essere semplice e intuitivo per favorire gli utenti meno avvezzi alla tecnologia.
